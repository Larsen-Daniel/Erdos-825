\documentclass[11pt]{amsart}
\usepackage{amsmath, amssymb, amsthm, mathtools,bbm}
\usepackage[margin=1in]{geometry}

\newtheorem{theorem}{Theorem}
\newtheorem{lemma}[theorem]{Lemma}
\newtheorem{corollary}[theorem]{Corollary}
\newtheorem{hypothesis}[theorem]{Hypothesis}

\def\Imag{\mathrm{Im}}
\def\Div{\mathrm{Div}^{*}}
\def\w{w}
\def\Z{\mathbb{Z}}
\def\N{\mathbb{N}}
\def\Q{\mathcal{Q}}
\def\B{\mathcal{B}}
\def\D{\mathcal{D}}
\def\L{\mathcal{L}}
\def\fin{\mathrm{fin}}

\def\BDP{B_{j-1}\cdot \Div(M_j)}

\begin{document}

\author{Daniel Larsen}
\title{Sufficiently abundant numbers are pseudoperfect}
\email{dlarsen@mit.edu}

\begin{abstract}
There exists an absolute constant $C$ such that if $n$ is any positive integer with $\sigma(n)\ge Cn$, then $n$ is a sum of distinct proper divisors.
This answers a question of Benkoski and Erd\H{o}s.
\end{abstract}

\maketitle

Following Sierpi\'nski \cite{S65}, we say a number $n$ is \emph{pseudoperfect} if it can be written as a sum of distinct proper divisors of $n$. 
We call  $\sigma(n)/n$, where $\sigma(n)$ is the sum of the divisors of $n$, the \emph{abundance index} of $n$.
Benkoski and Erd\H{o}s conjectured \cite{BE74} that if the abundance index of $n$ is sufficiently large, then $n$ must be pseudoperfect.
In this paper, we prove this conjecture.

Following a general strategy suggested by Tao and Bloom \cite{B825}, as exemplified in work of Croot \cite{C03}, Bloom \cite{B23}, and Conlon et al. \cite{C25},
we use a probabilistic method to show that a suitably weighted random selection process of proper divisors of $n$ produces $n$ as a sum with positive
probability. The main technical difficulty stems from the possibility that the prime factors of $n$ may accumulate on very different scales.
We use a greedy-but-not-too-greedy algorithm to approach $n$ from below closely enough that the divisors on one scale are suitable for ``terminal guidance.''

Following Friedman \cite{F93}, we find it convenient to recast the problem in terms of Egyptian fractions, representing $1$ as a sum of reciprocals of divisors of $n$.
Schematically, our approach can be decomposed into stages as follows:
\begin{enumerate}
\item[1.] We divide prime factors of $n$ into dyadic intervals, discarding intervals whose prime density is too low.
\vskip 5pt
\item[2.] We block the dyadic intervals into ranges of intervals such that there are no very large gaps within any block.
\vskip 5pt
\item[3.] We use a greedy algorithm to identify the first block on which the sum of reciprocals reaches $1$ with sufficient slack.
\vskip 5pt
\item[4.] We then pull back, if necessary, so that the next block of reciprocals add up to (ideally) twice the distance from the current sum to $1$.
\vskip 5pt
\item[5.] Finally, we use a circle method argument to show that the next block of reciprocals can precisely fill the gap. 
\end{enumerate}

The letter $p$ will always refer to a prime. We say $n$ is \emph{$N$-smooth} if all its prime factors are $\le N$ and \emph{$N$-rough} if all its prime factors
are $\ge N$.
We say something is true for almost all elements of a set $S$ if the number of exceptions has size $o(|S|)$. Finally, we write $A \cdot B = \{ab : a \in A, b \in B\}$.

We use standard asymptotic notation with the additions $x\sim y$, which means $x\in [y,2y)$,  and the equivalent statements $x \lesssim_\alpha y$ and $y \gtrsim_\alpha x$, which, for a parameter $\alpha > 1$ 
that will be specified prior to this usage, mean there exists an absolute constant $C$ such that $x <  \alpha^C \cdot y$.

We now state the main theorem.

\begin{theorem}
\label{main}
For every $\varepsilon > 0$ there exists an integer $L$ such that if $n$ is an integer with 
$$\sigma(n)/n > 2 + \varepsilon$$
and it has no prime factors less than $L$, then $n$ is pseudoperfect.
\end{theorem}

Because any positive integer multiple of a pseudoperfect number is pseudoperfect,
we may assume that $\sigma(n)/n = O(1)$.
We may also assume that $n$ is squarefree since if $\sigma(n)/n>2+\varepsilon$, the squarefree part of $n$ also has abundance index bounded away from $2$ if 
$$\sum_{p\ge L} \frac 1{p^2} = o(\varepsilon).$$
(We will assume that $L$ is sufficiently large in terms of $\varepsilon$ that all such claims are true.) We will also assume $\varepsilon$ is small, specifically that
$\varepsilon < 1$ and that 
\begin{equation}
\label{weird}
\left(1+\frac\varepsilon 8\right)\left(1-\frac\varepsilon{10}\left(1+\frac\varepsilon 8\right)\right)> 1 + \frac\varepsilon{100}.
\end{equation}

Define 
$$\w(A) = \sum_{1<a \in A} \frac{1}{a},$$
$\mathrm{Div}(N)$ to be the set of divisors of $N$, and $\Div(N)$ the set of divisors greater than $1$.  For $N$ squarefree and $L$-rough,
\begin{equation}
\label{L-rough}
\log (1+\w(\Div(N))) = \sum_{p|N} \log \frac{p+1}p = \sum_{p|N} \frac 1p + O\Bigl(\frac{1}{L}\Bigr).
\end{equation}
Let $x_1, \ldots, x_K$ be a maximal subsequence of the powers of $2$ such that $x_{i+1} \ge 2x_i$, and 
$$|P_i| > x_i/(\log x_i)^2$$ 
where
$$P_i = \{p \sim x_i : p \mid n\}.$$
Note that this sequence is non-empty since 
\begin{equation}\label{eq:contradiction}
\sum_{\substack{p \mid n \\ \nexists\, i : p \sim x_i}} \frac{1}{p} \ll \sum_{j \ge \lfloor \log L\rfloor} \frac{1}{j^2} \ll \frac{1}{\log L},
\end{equation}
which implies by \eqref{L-rough} that 
\begin{equation}
\label{unchanged}
\sum_{\substack{1<d\vert n\\ \exists p\not\in P: p\vert d}} \frac 1d \le
\frac{\sigma(n)}n \sum_{\substack{1<d\vert n\\ \forall p\vert d, \,p\not\in P}} \frac 1d \ll  
\frac 1{\log L} = o(\varepsilon),
\end{equation}
setting $P$ to be the union of the $P_i$.

Let $N_i = \prod_{p \in P_i} p$ and $J_i = [x_i^{8}, e^{x_i/\log^2 x_i}]$. Define an equivalence relation on $\{1, \ldots, K\}$ taking the transitive closure of the relation which identifies $i$ and $j$ when $J_i \cap J_j \neq \emptyset$. Note that when $J_i\cap J_j=\emptyset$ with $j>i$, we have $x_i\ll \log^{1+\varepsilon} x_j$.
There exist $a_1, \ldots, a_{r}$ with $a_1 = 1$ and $a_r = K + 1$ giving equivalence classes $S_j = [a_j, a_{j+1})$. 

Let $M_j = \prod_{i \in S_j} N_i$ and 
$$M_{j, \le i^*} = \prod_{\substack{i \in S_j \\ i \leq i^*}} N_i.$$
Let $B_j =  \mathrm{Div}(M_{\le j})$ where $M_{\le j} = \prod_{j'\le j} M_{j'}$. Note that the sets $B_{j-1}\cdot \Div(M_j)$ for $1\le j<r$ partition $\Div(M_{r-1})$.
By \eqref{unchanged},
\begin{equation}
\label{R of D}
\w(\Div(M_{\le r-1})) > 1+\frac\varepsilon2.
\end{equation}
Different values of $j$ correspond to exponentially separated scales on which $n$ has many prime factors. We want to apply the circle method to the divisors of $n$, but these vastly different scales will not combine well. Therefore we begin with a greedy process.

We initialize $A_0 = \{1\}$. Beginning with $j = 1$ we form $A_j$ according to the following rules. 
If 
$$\w(B_{j-1} \cdot \Div(M_j)) \ge \Bigl(1 + \frac{\varepsilon}{8}\Bigr)(1 - \w(A_{j-1}))$$
(which we think of as meaning the scales up to $j$ not only reach $1$ but do so decisively)
we set $A = A_{j-1}$, $B = B_{j-1}$, and $j_0 = j$ and stop. On the other hand, if 
\begin{equation}
\label{case}
\w(B_{j-1} \cdot \Div(M_j)) < \Bigl(1 + \frac{\varepsilon}{8}\Bigr)(1 - \w(A_{j-1}))
\end{equation}
then we start by partitioning the elements of $B_{j-1}\cdot \Div(M_j)$ into dyadic intervals and letting $E_j$ be obtained by choosing one element 
from each dyadic interval for which this is possible.
We then set $A_j = A_{j-1}\cup C_j$, where $C_j$ is a maximal subset of $B_{j-1} \cdot \Div(M_j)$ such that $\w(C_j) < 1 - \w(A_{j-1})$. 
If $\w(E_j)<1-\w(A_{j-1})$, we choose $C_j$ so that it contains $E_j$.
In fact, this is always possible since
$$\w(E_j) \ll \frac 1{x_{a_j}}= o(\w(\BDP)) \ll 1-\w(A_{j-1}),$$
using \eqref{case}.
Note that if
$$\w(A_{j-1}) + \w(B_{j-1}\cdot \Div(M_j)) < 1,$$
then we take $C_j$ to be all of $B_{j-1}\cdot \Div(M_j)$.

We claim that this process must terminate. The smallest elements of $B_{j-1}\cdot \Div(M_j)$ are all primes between $x_{a_j}$ and $2x_{a_j}$ which divide $n$. Therefore, no summand in 
$$\sum_{a\in B_{j-1}\cdot \Div(M_j)} \frac 1a$$
is greater than $\delta$ times the sum for any absolute constant $\delta>0$ when
$L$ is sufficiently large. 
When $C_j\neq B_{j-1}\cdot \Div(M_j)$, 
it follows that
\begin{equation}
\label{span}
\w(C_j)  = (1-o(1))(1-\w(A_{j-1})),
\end{equation}
so
\begin{equation}
\label{decay}
1-\w(A_j) = o(1-\w(A_{j-1}))
\end{equation}
and
\begin{equation}
\label{small}
\w((B_{j-1}\cdot \Div(M_j))\setminus C_j) \le (1+\frac\varepsilon 8-1+o(1))(1-\w(A_{j-1})) \le \frac\varepsilon 8+ o(1),
\end{equation}
subtracting \eqref{span} from \eqref{case}.
Thus, if the process does not terminate, then $\sum_j \w(C_j) < 1$ so remembering \eqref{R of D},
\begin{equation}
\label{untrue}
 \sum_j \w((B_{j-1}\cdot \Div(M_j))\setminus C_j) = \w(\Div(M_{\le r-1}))-\sum_j \w(C_j) \ge \frac\varepsilon 2.
 \end{equation}

Label the elements in $\{j: \w(A_{j-1}) + \w(B_{j-1}\cdot \Div(M_j)) \ge 1\}$ as $j_1<j_2<\cdots$. Then
$$\frac\varepsilon 2\le \sum_k \w((B_{j_k-1}\cdot \Div(M_{j_k}))\setminus C_{j_k}) <  2\w((B_{{j_1}-1}\cdot \Div(M_{j_1}))\setminus C_{j_1}) \le \frac \varepsilon 4+o(1)$$
by \eqref{untrue}, \eqref{decay}, and \eqref{small} respectively. This gives the desired contradiction.

Thus, we may assume that 
$$\w(B \cdot \Div(M_{j_0})) \ge (1 + \varepsilon/8)(1 - \w(A)).$$
We would like to use the circle method to find a subset of the summands of the left hand side which sum to $1-\w(A)$.
The difficulty is that $1-\w(A)$ may be very small at this point. We happily embraced this possibility before so that we could avoid the situation in which 
$$\w(A) + \w(B\cdot \Div(M_{j_0})) - 1$$
was very small. It is time to take a step backward from the greedy method and remove some elements of $A$.


Let $i_0$ be the smallest index such that
\begin{equation}
\label{index}
\w(D) \ge \Bigl(1 + \frac{\varepsilon}{8}\Bigr)(1 - \w(A)).
\end{equation}
Let $D = B \cdot \Div(M_{j_0,  \le i_0})$.
Note that $\sum_{p\mid N_{i_0}} 1/p = o(1)$. It follows from this and the fact that 
$$\w\Bigl(B \cdot \Div(M_{j_0,  \le i_0-1})\Bigr) < \Bigl(1 + \frac{\varepsilon}{8}\Bigr)(1 - \w(A))$$
that 
\begin{equation}
\label{controlled R}
\w(D) < 2-\w(A).
\end{equation}


\begin{lemma}
\label{1}
There exists a subset $D_0$ of $A$ such that setting 
$$\alpha = \frac{\w(D)}{1 - \w(D_0)},$$
we have
$$1 + \frac{\varepsilon}{100} \le \alpha \le \log^{1+\varepsilon} x_{a_{j_0}}.$$
\end{lemma}

\begin{proof}
If $w(D)\ge 1+\varepsilon/8$, it is clear by \eqref{controlled R} that we can take $D_0 = A$, and this will result in $\alpha < \frac 2{\varepsilon/8} = 16/\varepsilon$.
In the other direction, by \eqref{index},
$\alpha \ge 1+\frac\varepsilon8.$
This takes care of the $j_0=1$ case.

Now assume $j_0 > 1$. If $\w(D)\ll 1-\w(A)$, then we can take $D_0 = A$ and $1+\frac \varepsilon 8\le \alpha \ll 1$. Otherwise, 
$$1/\w(D)=o(1/(1-\w(A)))\le M_{\le j_0 - 1}.$$
Since every prime factor of this product has size at most $2x_{a_{j_0}-1},$
there exists some $d\in B$ such that $1/\w(D)$ is between $\varepsilon d/(40 x_{a_{j_0} - 1})$ and $\varepsilon d/20$. 
At the cost of  supposing only that $1/\w(D)$ is between $\varepsilon d/(80 x_{a_{j_0} - 1})$ and $\varepsilon d/10$, we may assume that
$$d\in \bigcup_{j<j_0} E_j\subset A.$$
Remove $d$ from $A$ to form $D_0$. Then
$$1 - \w(D_0) = 1 - \w(A) + \frac{1}{d} > \frac{1}{d} \gg \frac{\w(D)}{x_{a_{j_0} - 1}}.$$
Rearranging, $\alpha\ll x_{a_{j_0}-1} = o(\log^{1+\varepsilon} x_{a_{j_0}})$
as claimed. At the same time, we still have 
$$\alpha \ge  \frac{\w(D)}{1-\w(A)+\varepsilon \w(D)/10}.$$
Note that if $a/b \ge 1+\varepsilon/8$, then
$$\frac a{b+\varepsilon a/10} = \frac {a(b-\varepsilon a/10)}{b^2-\varepsilon^2 a^2/100} > \frac ab \bigl(1-\frac {\varepsilon a}{10b}\bigr) \ge 1+\frac\varepsilon{100}$$
by \eqref{weird}, unless $a/b\ge 5/\varepsilon$, in which case $a/(b+\varepsilon a/10) \ge 10/3$. Plugging in $a=\w(D)$
and $b=1-\w(A)$, we conclude that $\alpha \ge 1+\frac\varepsilon{100}$.

\end{proof}
This concludes the preliminary stage of the argument. The advantage of approaching $1$ within a specific block rather than proceeding one dyadic interval at a time
is that we may reserve judgment on which divisors from previous dyadic intervals should be used and simply throw everything within the block into the circle method.
We now prove a general theorem which will give what we need to finish the proof of Theorem~\ref{main}. To begin with we define the following condition:

\begin{hypothesis}[$\D\subset \N$, $y\in\N$, $0<\beta\le 1$] 
\label{H}
For every integer $h\in [y/2,y^{\lceil 10/\beta\rceil-2}]$, 
$$\sum_{d\in \D} h_d^2/d^2 > y^{1/4},$$
where $h_d$ is the distance between $h$ and $d\Z$,
\end{hypothesis}

This is the condition needed for the medium arcs in the following circle method argument.

\begin{theorem}
\label{circle}
Let $\ell/k\in (0,1]$, $\epsilon>0$ and $\beta>0$ be constants, and $\L$ be an integer sufficiently large in terms of $\epsilon$ and $\beta$.
Let $y_1,\ldots,y_s$ be a sequence of positive integers greater than $\L$
such that for every $i<s$, $y_i\le y_{i+1}/2$
and
$$\bigl[y_i^{\lceil 4/\beta\rceil-5/4},e^{y_i/\log^2 y_i}\bigr]\cap \bigl[y_{i+1}^{\lceil 4/\beta\rceil-5/4},e^{y_{i+1}/\log^2 y_{i+1}}\bigr] \neq \emptyset.$$
Let $\mathcal{Q}_i$ be a set of integers between $y_i$ and $2y_i$ such that the elements of $\mathcal{Q} := \bigcup_i \mathcal{Q}_i$ are pairwise
coprime. Let $\B$ be a set including $1$ of integers dividing $(y_1^2)!$ which are coprime to every element of $\Q$.
Let $E\subset \B$ be a subset which contains at the very least 
all elements of $\B$ less than $y_1^2$.
Let
$$\D = \Bigl(\B\cdot \mathrm{Div}(\prod_{q\in \Q} q)\Bigr)\setminus E.$$
Let $m$ be the least common multiple of $\D$. Assume that $k\mid m$.
Let $\alpha = w(\D)(\ell/k)^{-1}$.  If 
$$1+\frac\epsilon{100} \le \alpha \le \log^{1+\epsilon} y_1,$$
$$\frac{y_i^\beta}{\log^2 y_i} \ll |\mathcal{Q}_i|\le y_i^\beta$$ 
for all $i$,
and Hypothesis~\ref{H} holds for $y=y_1$, then there exist at least two subsets $\D'$ of $\D$ such that $w(\D') \equiv \ell/k\pmod 1$.
\end{theorem}

\begin{proof}

Let $F(\D)$ be the number of solutions to 
\begin{equation}\label{eq:cong}
\sum_{d \in \D'} \frac{1}{d} \equiv \frac{\ell}{k} \pmod{1},
\end{equation}
where $\D' \subset \D$, and let $F_\alpha(\D)$ be the weighted number of solutions
giving $\D'$ satisfying \eqref{eq:cong} weight $(\alpha-1)^{|\D|-|\D'|}$.


Setting $e(x) =e^{2\pi i x}$, by orthogonality of characters on $\Z/m\Z$,
\begin{align*}
F_\alpha(\D) &= \sum_{\D'\subset \D}\mathbbm{1}_{w(\D')\equiv \ell/k \ \text{mod } 1} (\alpha-1)^{|\D|-|\D'|}\\
&=\sum_{\D'\subset \D}   \frac 1m \sum_{-m/2 < h \leq m/2} e(h(w(\D')-\ell/k))(\alpha-1)^{|\D|-|\D'|}\\
&= \frac{1}m \sum_{-m/2 < h \leq m/2} e(-h\ell/k) \prod_{d \in \D} \Bigl(\alpha - 1 + e(h/d)\Bigr).
\end{align*}
Moreover, 
$$F(\D) \ge \frac{F_\alpha(\D)}{\max(1,\alpha - 1)^{|\D|}}.$$
We write
$$F_\alpha(\D) = \frac{\alpha^{|\D|}}m \sum_{-m/2 < h \leq m/2} e\bigl(h(\w(\D)/\alpha - \ell/k)\bigr) f(h)
=  \frac{\alpha^{|\D|}}m \sum_{-m/2 < h \leq m/2} f(h)$$
where 
$$f(h) = \prod_{d \in \D} \left(\frac{(\alpha - 1)e(-h/d\alpha) + e\bigl(\frac{(\alpha-1)h}{d\alpha}\bigr)}{\alpha}\right).$$
It is clear that $h = 0$ contributes $\alpha^{|\D|}/m$ to $F_\alpha(\D)$. We now need to show the other components of the sum do not have much of an effect.

Let $t = \lceil10/\beta\rceil$.
Fix $i \le s$.
Let $h_d \in (-d/2, d/2]$ denote the residue of $h$ mod $d$. 
Suppose it is not true that $|h_d| \leq y_i^{t-2}$ for almost all $d$ which can be expressed as a product of $t$ elements of $\mathcal{Q}_i$. Then 
$$|h_d|/d \gg y_i^{-2}$$
for $\gg y_i^{10} / \log^{2t} y_i$ values of $d\in\D$. Note that when $d/|h_d|$ is greater than some fixed power of $\alpha$,
$$1 - \Bigl| \frac{(\alpha - 1)e(-h/d\alpha) + e\bigl((\alpha - 1)h/d\alpha\bigr)}{\alpha}\Bigr| \gtrsim_\alpha \frac{h_d^2}{d^2}.$$
Thus for such $h$, $|f(h)| < \exp(-y_i^{5})$ for large $y_i$. We will use this method of bounding $f(h)$ by bounding $|h_d|/d$ repeatedly.

If $|h_d| \leq y_i^{t-2}$ for almost all considered $d$, there exist $q_1,\ldots,q_{t-1} \in \mathcal{Q}_i$ such that for 
almost all $q_t \in \mathcal{Q}_i$, setting $d = q_1 \cdots q_t$, $|h_d| \leq y_i^{t-2}$, i.e.\ $h$ is within $y_i^{t-2}$ of a multiple of $q_1\cdots q_t$, which we call $M$. We see that $M$ must be divisible by almost all elements of $\mathcal{Q}_i$, since $q_1\cdots q_{t-1} > y_i^{t-2}$.

Fix $q_t\in \mathcal{Q}\cap (1,2y_i)$. Either $q_t \mid M$ or for almost all $q_1,\ldots,q_{t-1} \in \mathcal{Q}_i$, setting $d = q_1\cdots q_t$, $|h_d|/d \gg 1/q_t$, implying that 
$$|f(h)| < \exp(-y_i^{5}).$$
Thus either $h$ is within $y_i^{t-2}$ of a multiple of 
$$R_i := \prod_{\substack{q \in \mathcal{Q} \\ q \leq 2y_i}} q,$$
or its contribution is exponentially small.
In particular,
$$\sum_{|h| \in [y_{i}^{t-2}, R_i/2]} |f(h)| < \exp(-y_i^{5}/2)$$
because 
$$R_i\le (y_1^2)!\prod_{i'\le i}(2y_{i'})^{y_{i'}^\beta} = (2y_i)^{O(y_i^2)}.$$
Since these intervals of summation overlap as we vary $i$,
$$\sum_{y_{1}^{t-2} \leq |h| \leq m/2} |f(h)| < 2\exp(-y_{1}^{5}/2).$$

Now consider the case that $0<|h| \le y_{1}^{1-5\beta/12}$. For any $d$,
$$\Bigl|\Imag \frac{(\alpha - 1)e(-h/d\alpha) + e((\alpha - 1)h/d\alpha)}{\alpha}\Bigr| \lesssim_\alpha \frac{h^3}{d^3},$$
Let
$$\zeta_{\Q}(\sigma) = \prod_{q\in \Q}(1+q^{-\sigma}) = \frac{\sum_{d\in \D} d^{-\sigma} + \sum_{d\in E} d^{-\sigma}}{\sum_{b\in \B} b^{-\sigma}}.$$
Then
$$\log \zeta_{\Q}(3) \le  \sum_{q\in\Q} q^{-3} = \sum_i \sum_{q\in \Q_i} q^{-3} \le \sum_i  y_i^{\beta-3} \le 2y_1^{\beta-3}$$ 
so
$$\zeta_{\Q}(3) - 1 \ll y_1^{\beta-3}.$$
Consequently,
\begin{align*}
|\arg f(h)| \lesssim_\alpha h^3\sum_{d\in \D}d^{-3} 
&\le h^3 \Bigl(\sum_{b\in \B} b^{-3} e^{2y_1^{\beta-3}} - \sum_{d\in E} d^{-3}\Bigr)
\ll h^3 \Bigl(\sum_{b\in\B\cap [y_1^2,\infty)}b^{-3}+\sum_{b\in \B} b^{-3} y_1^{\beta-3}\Bigr)\\
& \ll h^3y_1^{\beta-3}\sum_{b\in \B}b^{-3} + h^3 y_1^{-4} \ll y_1^{-\beta/4},
\end{align*}
implying that $\operatorname{Re} f(h) > 0$.

Suppose $y_1^{1-5\beta/12}\le |h| \le y_1/2$. Then for all  $q\in \Q_1$, $|h_q|/q \gg y_1^{-5\beta/12}$. Therefore, the total contribution from this range is at most 
$\exp(-y_1^{\beta/12})$.  The remaining range of $h$-values is covered by Hypothesis~\ref{H}.

Putting it all together,
$$F(\D) \geq \frac{\alpha^{|\D|}}{m(\max(1,\alpha - 1))^{|\D|}}(1 - o(1)) > 1,$$
using the fact that $|\D| > 2^{y_s/\log^3 y_s}$ and
$$\frac\alpha{\max(1,\alpha - 1)} > 1+\frac 1{\log^{1+\epsilon} y_1}.$$
\end{proof}

We claim Hypothesis~\ref{H} holds for $\D = B \cdot \Div(M_{j_0,  \le i_0})$, $y=x_{a_{j_0}}$, and $\beta = 1$.
Indeed, if $h$ is fixed, $h + C$ can have $O(1)$ factors in $P_{a_{j_0}}$ when $|C|\le x_{a_{j_0}}^{2/3}$.
Consequently, almost all primes in $P_{a_{j_0}}$ will not have a multiple within $x_{a_{j_0}}^{2/3}$ of $h$, from which the hypothesis follows.

We can therefore apply Theorem~\ref{circle} to $\B = B$, $\Q_i = P_{i+a_{j_0}-1}$ for $i=1,\ldots,i_0-a_{j_0}+1$,
$$\D = B \cdot \Div(M_{j_0,  \le i_0}),$$
$\beta=1$, $\epsilon = \varepsilon$, $\L=L$, and $\ell/k = 1-\w(D_0)$.
Let $D'$ be a non-empty subset of $D$ with 
\begin{equation}
\label{cong}
\w(D')\equiv  1 - \w(D_0)\pmod 1.
\end{equation}
By \eqref{controlled R}, 
$$\w(D') \le w(D) < 2-w(A)\le 2-w(D_0),$$
so in fact the two sides of \eqref{cong} are equal.
Thus
$$n = \sum_{\substack{d\in D'\cup D_0\\ d>1}} \frac{n}{d}$$
as elements of $D'$ must have a prime factor $\ge y_{a_{j_0}}$, distinguishing them from elements of $D_0$.
This concludes the proof of Theorem~\ref{main}.

\begin{corollary}
There exists an absolute constant $C$ such that if $n$ is any positive integer with $\sigma(n)\ge Cn$, then $n$ is a sum of distinct proper divisors.
\end{corollary}

\begin{proof}
Fix $\varepsilon = 1/10$, so \eqref{weird} holds. Let $L$ be the corresponding value guaranteed by Theorem~\ref{main}. If $n$ is not pseudoperfect, then no factor of $n$ is pseudoperfect. Let $m$ be the $L$-rough part of $n$. Then
$$\frac{\sigma(n)}n \le \frac{\sigma(m)}m \prod_{p< L} \frac p{p-1} \ll 1$$
if $m$ is not pseudoperfect.
\end{proof}

We acknowledge the assistance of Claude Opus 4.5 and ChatGPT 5.2 Pro for proofreading.

\begin{thebibliography}{9}
\bibitem{BE74}Benkoski, S. J. and Erd\H{o}s, P., On weird and pseudoperfect numbers. \textit{Math.\ Comp.} \textbf{28} (1974), 617--623.

\bibitem{B23}Bloom, T.~F.,
On a density conjecture about unit fractions. With an appendix by Bloom and Bhavik Mehta. 
\textit{J.\ Eur.\ Math.\ Soc.} (JEMS) \textbf{27} (2025), no.\ 11, 4563--4589.

\bibitem{B825}Bloom, T.~F., Erd\H{o}s Problem \#825, \texttt{https://www.erdosproblems.com/825}, accessed 2026-01-29.

\bibitem{C25}Conlon, D., Fox, J., He, X., Mubayi, D., Pham, H.~T., Suk, A., Verstra\"ete, J.,
 A question of Erd\H{o}s and Graham on Egyptian fractions, \emph{Discrete Analysis}, 2025:28.
 
\bibitem{C03}Croot, E. S., III, On a coloring conjecture about unit fractions. 
\textit{Ann.\ of Math.} (2) \textbf{157} (2003), no.\ 2, 545--556.

\bibitem{F93}Friedman, C.~N.
Sums of divisors and Egyptian fractions.
\textit{J.\ Number Theory} \textbf{44} (1993), no.\ 3, 328--339.

\bibitem{S65}Sierpi\'nski, W. Sur les nombres pseudoparfaits. \textit{Mat.\ Vesnik} \textbf{2} (17) (1965), 212--213. 

\end{thebibliography}
\end{document}